\begin{tabular}%
       {|>{\colleft}p{2.3cm}%
        |>{\colleft}p{9cm}|}
\hline
{\bf Project Management} &
{\bf Model State Planning Worksheet PM-2} \\
\hline
\hline
{\bf Attribute} & {\bf Description} \\
\hline
\hline
{\sc Model name} &
One of the {\ck} models: organization, task, agent,
knowledge, communication, design model. \\
\hline
{\sc State variable} &
A part or component(s) of the selected model on which project
work is to done (e.g., the inference layer of the knowledge model). \\
\hline
{\sc State value} &
An indicator of the degree of completion to be achieved by the
work on the selected model component(s). The following qualitative
five-point range is useful: \\
 &
{\bf 1. Empty:} The starting state value, indicating that no work
has been done yet. \\
 &
{\bf 2. Identified:} Basic features relating to the selected model
component(s) have been listed. These may refer to essential
characteristics of the model component (e.g., the task decomposition
shows the typical features of an assessment type of task), identifying
external requirements and inputs (e.g., listing the information sources
that will be used for the work on the model). \\
 &
{\bf 3. Described:} The modelling or implementation work has
been fully carried out. This is the level of a complete first
version or draft. \\
 &
{\bf 4. Validated:} The work done is tested, verified, and validated
with respect to outside criteria or sources (e.g., against given
quality measures, external requirements, or by checking the
correctness of developed models with relevant experts). \\
 &
{\bf 5. Completed:} The work on the model component
is finished according to the
established acceptance criteria (e.g., being accepted and signed off
after a review with the client). \\
\hline
{\sc Quality metrics} &
The quality metrics according to the quality plan that will be used to
measure whether the desired model state has indeed been achieved.
Also, the procedure to establish this is to be indicated here. \\
\hline
{\sc Role} &
This is an optional attribute of a model state. It can be used to
indicate that a model state plays a
specific role in a project, e.g., as a milestone at which a
go/no-go decision is to be taken.  \\
\hline
{\sc Dependencies} &
This is an optional attribute: sometimes it is useful to
indicate that achievement of a model state critically depends
on certain external inputs (e.g., a management decision to be taken,
equipment to be available, or results from another part of the
project to be finished). \\
\hline
\end{tabular}
