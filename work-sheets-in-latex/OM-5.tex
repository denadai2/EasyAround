\begin{tabular}%
       {|p{3cm}%
        |p{9.5cm}|}
\hline
{\bf Organization Model} &
  {\bf Checklist for Feasibility Decision Document: Worksheet OM-5} \\
\hline
\hline
\sc Business feasibility &
   {\rm  \textbf{Benefits}: the itinerary process is quicker, the client is more satisfied, travel agents can schedule their work time on a higher number of customers;} \\
 & {\rm \textbf{Added value}: the speed up should be quite significant, it is expected that the TA can satisfy a client surplus of 30\% with the time he saved in building and reviewing degigns.} \\
 & {\rm \textbf{Costs}: the costs are a summation of the salary of the employees working on building the software (programmers, experts) and the time spent in integrating the licenced content into the automated system;} \\
 & {\rm \textbf{Organizational Changes}: the system is built to avoid organizational changes.} \\
 & {\rm \textbf{Risks}: the system could have difficulties in selecting the right locations based on customer's requests, not posing as an advantage to the Travel Agent. In this case the workload would not decrease.} \\
\hline
\sc Technical feasibility &
   {\rm \textbf{Complexity}: the complexity level of the required reasoning is high, because it need the integration of a lot of informal knowledge into a formal system, and the handling of many constraints;} \\
 & {\rm \textbf{Critical aspects}: the solution must be developed correctly, otherwise the risk of losing clients grows. Furthermore, if the results are not as expected, the software could not be accepted or used inside the acency.} \\
 & {\rm \textbf{Success Measures}: if the design is coherent with the requirements, if there are no constraint violations, if it corresponds to the preferences of the client, and it is at least the same or better than a manual design done by the TA, then it is a success.} \\
 & {\rm \textbf{User Interface}: the UI can be constructed to be very simple and intuitive, requiring no additional knowledge about IT systems from the user. } \\
 & {\rm \textbf{Additional Interactions}: the only extern interaction is with the structured database of locations, which basic structure is fully impemented and documented in many shapes and programming languages.} \\
& {\rm \textbf{Further technological risks}: there are no further risks;} \\
\hline
Project feasibility
 & {\rm \textbf{Commitment}: the TAs are interested in a mechanism that allows them to save time for single-customer itinerary design, the president is interested in employing new technologies to increment profit.} \\
 & {\rm \textbf{Resources}: since the expertise is provided by the agency itself, the necessary resources left are the ones needed for the programmers. Being freelancers, their cost is relatively limited by the absence of an organization that coordinates the work.} \\
 & {\rm \textbf{Knowledge}: the knowledge is available since it's provided by the agency itself, and it's largely available on public means such as the web;} \\
 & {\rm \textbf{Expectations}: the expectation are realistic;} \\
 & {\rm \textbf{Communication}: the communication is efficient, both between the programmers who have worked with each other previously, and between the expert consultant and the team since they are acquaintances.} \\
\hline
Proposed actions
 & {\rm 1. {\em Focus:} speed-up of the design process, increased number of customers;} \\
 & {\rm 2. {\em Target solution:} Automatization of the design and revision process;} \\
 & {\rm 3. {\em Results, costs, and benefits:} satisfaction of the client, saved workload and working time for the TA;} \\
 & {\rm 4. {\em Project actions}: building the Knowledge Model, create the Design Model, create the Communication Model, implement the system, embed the knowledge in the software, test the software and collect results;} \\
 & {\rm 5. {\em Risks:} the system could have difficulties in selecting the right locations based on customer's requests, not posing as an advantage to the Travel Agent. In this case the workload would not decrease} \\
\hline
\end{tabular}





