\begin{tabular}%
       {|>{\colleft}p{3cm}%
        |>{\colleft}p{8.5cm}|}
\hline
{\bf Organization Model} &
   {\bf Variant Aspects Worksheet OM-2} \\
\hline
\hline
\sc Structure &
   {\rm
   Give an organization chart of the considered (part of the) organization
   in terms of its departments, groups, units, sections, ...
   } \\
\hline
\sc Process &
   {\rm
   Sketch the layout (e.g., with the help of a UML activity diagram) 
   of the business
   process at hand. A process is the relevant part of the value
   chain that is focused upon. A process is
   decomposed into tasks, which are detailed in worksheet OM-3.
   } \\
\hline
\sc People &
   {\rm
   Indicate which staff members are involved, as actors or
   stakeholders, including decision makers, providers, users or
   beneficiaries (``customers'') of knowledge. These people do not need
   to be actual people, but can be functional roles played by people in
   the organization (e.g., director, consultant)
   } \\
\hline
\sc Resources &
   {\rm
   Describe the resources that are utilized for the business
   process. These may cover different types, such as:
   } \\
 & {\rm 1. Information systems and other computing resources} \\
 & {\rm 2. Equipment and materials} \\
 & {\rm 3. Technology, patents, rights} \\
\hline
\sc Knowledge &
   {\rm
   Knowledge represents a special resource exploited in a business
   process. Because of its key importance in the present context, it
   is set apart here. The description of this component of the
   organization model is given separately, in worksheet OM-4 on
   knowledge assets.
   } \\
\hline
\sc Culture \& power &
   {\rm
   Pay attention to the unwritten rules of the game,
   including styles of working and communicating (``the way we do
   things around here''), related social and interpersonal
   (nonknowledge) skills, and formal as well as informal    
   relationships and networks.
   } \\
\hline
\end{tabular}
