\begin{tabular}%
       {|>{\colleft}p{3cm}%
        |>{\colleft}p{8cm}|}
\hline
  \bf Organization, Task, Agent Models &
  {\bf  Worksheet OTA-1: Checklist for Impact and Improvement Decision Document
  } \\
\hline
\hline
\sc Impacts and changes in organization &
   {\rm
   Describe which impacts and changes the considered knowledge system
   solution brings with respect to the organization, by comparing the
   differences between the organization model (worksheet OM-2) in the
   current situation, and how it will look in the future. This
   has to be done for all (variant) components in a global fashion
   (specific aspects for individual tasks or staff members are
   dealt with below).
   } \\
 & {\rm 1. Structure
           } \\
 & {\rm 2. Process
           } \\
 & {\rm 3. Resources
           } \\
 & {\rm 4. People
           } \\
 & {\rm 5. Knowledge
           } \\
 & {\rm 6. Culture \& power
           } \\
\hline
\sc Task/agent-specific impacts and changes &
   {\rm
   Describe which impacts and changes the considered knowledge system
   solution brings with respect to individual tasks and agents, by
   comparing the differences between the task and agent models
   (worksheets TA-1/2 and AM-1) in the current situation, and what they
   will look like in the future. It is important to look not only
   at the staff members directly involved in a task but also
   other actors and stakeholders (decision-makers, users, clients).
   } \\
 & {\rm 1. Changes in task layout} \\
 & {\rm (flow, dependencies, objects handled, timing, control)
           } \\
 & {\rm 2. Changes in needed resources
           } \\
 & {\rm 3. Performance and quality criteria
           } \\
 & {\rm 4. Changes in staffing, involved agents
           } \\
 & {\rm 5. Changes in individual positions, responsibilities,
           authority, constraints in task execution
           } \\
 & {\rm 6. Changes required in knowledge and competences
           } \\
 & {\rm 7. Changes in communication
           } \\
\hline
\sc Attitudes and commitments &
   {\rm
   Consider how the individual actors and stakeholders involved will
   react to the suggested changes, and whether there will be a
   sufficient basis to successfully carry through these changes
   } \\
\hline
\sc Proposed Actions &
   {\rm
   This is the part of the impacts and improvements decision document
   that is directly subject to managerial commitment and
   decision-making. 
   It weights and integrates the previous analysis results
   into recommended concrete steps for action:
   } \\
 & {\rm 1. {\em Improvements:} What are the recommended changes, with
           respect to the organization, as well as individual tasks,
           staff members, and systems?
           } \\
 & {\rm 2. {\em Accompanying measures:}
           What supporting measures are to be taken to facilitate
           these changes (e.g., training, facilities)
           } \\
 & {\rm 3. What further {\em project action} is recommended
           with respect to the undertaken knowledge system solution?
           } \\
 & 4.      {\em Expected results, costs, benefits}: reconsider items from
           the earlier feasibility decision document
           \\
 & {\rm 5. If circumstances inside or outside the organization change,
           under what {\em conditions} is it wise to reconsider the
           proposed decisions?
           } \\
\hline
\end{tabular}
