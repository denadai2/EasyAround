\begin{tabular}{%
       |>{\colleft}p{3cm}%
       |>{\colleft}p{8.5cm}|}
\hline
{\bf Communication model} &
   {\bf Transaction Description Worksheet CM-1} \\
\hline
\hline
\sc Transaction identifier/name &
   {\rm
   A transaction is to be defined for each information object that is
   output from some leaf task in the task model or in the knowledge
   model (i.e., a transfer function), and that must be communicated to
   another agent for use in its own tasks. The name must reflect, in a
   user-understandable way, what is done with this information object
   by the transaction. In addition to the name, give a brief
   explanation here of the purpose of the transaction.
   } \\
\hline
\sc Information object &
   {\rm
   Indicate the (core) information object, and between which two
   tasks it is to be transmitted.
   } \\
\hline
\sc Agents involved &
   {\rm
   Indicate the agent that is sender of the information object,
   and the agent that is receiving it.
   } \\
\hline
\sc Communication plan &
   {\rm
   Indicate the communication plan of which this transaction is a
   component.
   } \\
\hline
\sc Constraints &
   {\rm
   Specify the requirements and (pre)conditions that must be fulfilled
   so that the transaction can be carried out. Sometimes, it is also
   useful to state post-conditions that are assumed to be valid after
   the transaction.
   } \\
\hline
\sc Information exchange specification &
   {\rm
   Transactions can have an internal structure, in that they consist
   of several messages of different types, and/or handle additional
   supporting information objects such as explanation or help items.
   This is detailed in worksheet CM-2. At this point, only a reference
   or pointer needs to be given to a later info exchange spec.
   } \\
\hline
\end{tabular}
