\begin{tabular}{%
	|>{\colleft}p{3.5cm}%
	|>{\colleft}p{8cm}|}
\hline
\bf Knowledge Model 	& \bf Worksheet KM-1: Checklist Knowledge-Model
Documentation Document \\
\hline\hline
\bf Document entry	& \bf Description \\ 
\hline
\sc Knowledge model	& 
Full knowledge-model specification in text
plus selected figures.
\\ \hline
\sc Information sources used &
Listing of all the information sources about the application domain
that were consulted. This list is first produced during the identification
stage.
\\ \hline
\sc Glossary & 
Listing of application-domain terms together with a definition, in
textual form or other. Using Internet technology, one can create a
glossary with hyperlinks to text and pictures that explains the terms. 
\\ \hline
\sc Components considered &
List of potentially reusable components that were considered in the
identification stage, plus a decision and a rationale for why the
component was or was not used. The components are typically of two
types: task-oriented (e.g., task templates) and domain-oriented
(e.g., ontologies, knowledge bases). 
\\ \hline
\sc Scenarios &
A list of the scenarios for solving application problems collected
during the model-construction process. 
\\ \hline
\sc Validation results & 
Description of the result of validation studies, in particular
paper-based simulation and/or computer simulations (prototyping).
\\ \hline
\sc Elicitation material &
Include material gathered during elicitation activities (e.g., interview
transcripts) in appendices.
\\ \hline
\end{tabular}
