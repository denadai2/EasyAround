\begin{tabular}{ %
       |>{\colleft}p{3cm}%
       |>{\colleft}p{8.5cm}|}
\hline
{\bf Communication model} &
   {\bf Information Exchange Specification Worksheet CM-2} \\
\hline
\hline
\sc Transaction &
   {\em
   Give the transaction identifier and the name of which this information
   exchange specification is a part.
   } \\
\hline
\sc Agents involved &
1. {\bf Sender}; agent sending the information item(s)
\newline
2. {\bf Receiver}: agent receiving the information item(s)
\\
\hline
\sc Information items &
   List all information items that are to be transmitted in this
   transaction. This includes the (`core') information object the
   transfer of which is the purpose of the transaction. However, it may contain
   other, supporting, information items, that, for example, provide help
   or explanation. For each information item, describe the following:
   \\
&  1. {\bf Role}: whether it is a {\em core} object, or a {\em
      support} item.
   \\
&  2. {\bf Form}: the syntactic form in which it transmitted to
      another agent , e.g., data string, canned text, a certain type of
      diagram, 2D or 3D plot.
   \\
&  3. {\bf Medium}: the medium through which it is handled in the
      agent-agent interaction, e.g., a pop-up window, navigation and
      selection within a menu, command-line interface, human
      intervention.
   \\
\hline
\sc Message specifications &
   Describe all messages that make up the transaction. For each
   individual message describe:
   \\
&  1. {\bf Communication type}: the communication type of the
      message describing its intention (``illocutionary force,'' in
      speech-act terminology). 
   \\
&  2. {\bf Content}: the statement or proposition contained in the
      message.
   \\
&  3. {\bf Reference}: in certain cases, it may be useful to add
      a reference to, for example, what domain knowledge model or
      agent capability is required to be able to send or process the
      message.
   \\
\hline
\sc Control over messages &
   Give, if necessary, a control specification over the messages
   within the transaction. This can be done in pseudocode format or
   in a state-transition diagram, similar to how the control over
   transaction within the communication plan is specified. The
   difference is just the level of detail.
   \\
\hline
\end{tabular}
